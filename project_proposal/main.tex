\documentclass[a4paper, 12pt]{article}

% import packages
\usepackage{authblk}

% title config
\title{Motor based microtubule sliding forces in axon growth cones}
\author{Calvin Sprouse}
\affil{PHYS 322, Winter 2024}
\date{2024 Februrary 27}

% bibliography config
\bibliographystyle{abbrv}

\begin{document}
\maketitle

\section*{Introduction}
% from cell to neuron to axon
Living cells house a complex internal symphony of functions mediated by proteins. These functions are preformed on the background of the cytoskeleton, a dynamic weave of filaments guided and locked by force generating proteins providing structure to the cell. The neuron is just one class of living cell adapted to function as a network signal transceiver or node. Nodes require connections and for this neurons develop two main protrusions from the cell body beginning as neurites. Most neurites become dendrites; these smaller branches support bidirectional transport of both signal and cargo. They also serve as targets for incoming network connections from other neurons. One neurite on each neuron becomes the axon; this longer branch is adapted to support outbound signaling from the cell body. The extent of neuronal protrusion varies by function over a magnitude scale of micrometers to meters. Regardless of distance, the axon grows following chemical cues and reaches out to dendrites forming the neuronal network.

% from axon to internal microtubule structure to growth cone
The axon, being uniquely designed for signal propagation, expresses uni\-que traits in the neuron context. Most notably is the emergent ordering of microtubule filaments in the axon. Microtubules are dynamic assemblies of tubulin and serve as molecular highways. The plus-end of a microtubule (mt) is the end where polymerization activity is overwhelmingly prevalent. The polarity of an mt is its relative orientation in the context of the cell body; if an mts plus-end is further from the cell body than the minus-end, where polymerization activity tends not to occur, that mt is plus-end-out. This leads to a natural road analogy wherein mts are one-way roads for vehicles. The vehicles in this analogy are molecular motors which are described as plus-directed or minus-directed based on which end of the mt they tend to walk towards. The polarity pattern then describes the spacial distribution of mt polarity and indicates the function of the region. It follows that dendrites express roughly even distributions of plus-end-out and minus-end-out mts. Axons, being outbound signal pathways, express a nearly exclusive plus-end-out pattern.

% but what does the growth cone do


\section*{Model}



% bibliography
\bibliography{axon_growth_project}
\end{document}
