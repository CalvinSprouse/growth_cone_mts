\documentclass{beamer}

% import packages
\usepackage[T1]{fontenc}
\usepackage{lmodern}
\usepackage[a0paper]{beamerposter}
\usepackage{graphicx}
\usepackage{booktabs}
\usepackage{tikz}
\usepackage{pgfplots}
\usepackage{anyfontsize}
\usepackage{siunitx}

% configure pgfplots
\pgfplotsset{compat=1.14}

% specify theme and color theme
\usetheme{gemini}
\usecolortheme{earth}

% define poster lengths for columns
% If you have N columns, choose \sepwidth and \colwidth such that
% (N+1)*\sepwidth + N*\colwidth = \paperwidth
\newlength{\sepwidth}
\newlength{\colwidth}
\setlength{\sepwidth}{0.025\paperwidth}
\setlength{\colwidth}{0.3\paperwidth}
% define the separator column command
\newcommand{\separatorcolumn}{\begin{column}{\sepwidth}\end{column}}

% configure the author block
\title{Microtubule force generation in axon growth cones}
\author{Calvin Sprouse}
\institute[CWU]{Department of Physics, Central Washington University}

% configure the footer
\footercontent{
    PHYS 322 Computational Biophysics \hfill
    2024 March 12 \hfill
    \href{mailto:calvin.sprouse@cwu.edu}{calvin.sprouse@cwu.edu}
}

% setup logo
\logoright{\includegraphics[height=7cm]{figures/logos/cwu_logo.png}}


%-----------------------------------------------------------------------------%
% begin poster
\begin{document}

% begin the content framework
\begin{frame}[t]
\begin{columns}[t]
\separatorcolumn%


%-----------------------------------------------------------------------------%
% first column of content
\begin{column}{\colwidth}

%------------------------------------------------
% introduction/abstract block
\begin{block}{Introduction}

The neuron is a class of living cell adapted to function as a network node. Neurons have two types of protrusion from their cell body which begin as neurites. Most neurites become dendrites, which are smaller branches that support bidirectional transport of both signal and cargo. They also serve as targets for incoming network connections from other neurons. One neurite on each neuron becomes the axon. This longer branch is adapted to support outbound signaling from the cell body. While the dendrites remain relatively close to cell body the axon grows following chemical cues. The rate of axon growth varies by the function of the neuron and can extend from micrometers to meters.

Besides its length, the axon expresses unique internal traits from the dendrite. Most notably is the emergent ordering of microtubule (MT) filaments in the axon~\cite{nedelec1997n13}. MTs are dynamic assemblies of tubulin that serve as molecular highways. The plus-end of an MT is where most polymerization activity occurs. The polarity of an MT is its relative orientation in the context of the cell body. Molecular motors are sensitive to polarity as each motor has a directional preference. Dynein, for example, is a minus-directed motor and walks away from the plus-end. Kinesin, on the other hand, is a plus-directed motor. The polarity pattern describes the spatial distribution of MT polarity and indicates the function of the region.

\begin{figure}
    \centering
    \includegraphics[width=0.5\textwidth]{figures/background/neuron_with_mts.png}
    \caption{\label{fig:neuron_background}
        A neuron with dendrites and an axon extending from the cell body. MTs are shown in two regions: left is a dendrite, and right is an axon. The plus-end of the MTs is indicated by a black rectangle. The orientation of the MTs with respect to the cell body is indicated by the color: blue MTs are plus-end-out, orange MTs are minus-end-out.}
\end{figure}

The axon shown in Figure~\ref{fig:neuron_background} expresses the typical plus-end-out polarity pattern whereas the dendrite exhibits a typical mixed polarity pattern.

\end{block}

\begin{block}{Background}

% representative figure of GC

Axon growth is lead by a region called the growth cone (GC). The GC is distinguished from the rest of the axon by a decrease in MT density, sensing equipment for reading chemical guidance cues, and a force generating network of actin filaments and myosin motors.

% figure of actomyosin network and treadmill

The actomyosin network is a treadmill of actin filaments and myosin motors thought to be the primary diver in elongation~\cite{craig2012bj}. The treadmill behavior emerges from the combination of leading edge polymerization and trailing edge depolymerization of the actin. Myosin acts simultaneously as a crosslinker for the network and the transporter of trailing edge depolymerize actin filament to the leading edge for polymerization. By modulating actin adhesion to the substrate the GC can execute carefully controlled growth and steering. The role of MTs is currently understood as signal pathways between the chemical sensing equipment and the actomyosin network~\cite{craig2012bj, kalil2005con, sanchez-huertas2021fmn}.

% figure of MT penetration into GC region over actomyosin network

We investigate the role of the MTs in the GC regions as conveyors of force from the axon. In the axon, Dynein motors act between MTs inducing \(\qty{}{\pico\newton}\) scale forces primarily in the direction of the MTs plus-end~\cite{nedelec1997n13}. The forces acting on the GC are observed to be small, on the order of \(\qty{}{\pico\newton}\)~\cite{devincentiis2020jn, raffa2023scdb}. Furthermore, it seems the MTs protruding into the GC act to stabilize against contractile forces generated in the actomyosin network and the GC membrane~\cite{raffa2023scdb}. We extend the work done in Ref.~\cite{craig2012bj} to include distill-end MT force generation by a Dynein sorting mechanism and investigate the consequence of interaction in the GC.

\end{block}

\end{column}
\separatorcolumn%


%-----------------------------------------------------------------------------%
% second column of content
\begin{column}{\colwidth}

%------------------------------------------------
% content block
\begin{block}{Model}

Model, modifications to the base model

\end{block}

\end{column}
\separatorcolumn%


%-----------------------------------------------------------------------------%
% third column of content
\begin{column}{\colwidth}

%------------------------------------------------
% content block
\begin{block}{Results}

Results, what did this tell us.

\end{block}

%------------------------------------------------
% future work block
\begin{block}{Future Work}

This model is constructed on a population based steady state model with temporal attributes included as an after-thought. A more rigorous approach would likely include an agent based model with mt motions, protein binding events, and actin network activity taking place with defined rates. An agent based simulation constructed in this way could explore more precise relationships between actin treadmilling, adhesion, and mt force generation by sliding and polymerization. Constructed properly, this agent based simulation could explore the two-dimensional landscape of the growth cone and investigate the role of mt force generation in growth cone guidance or lack thereof. Furthermore, such a model would have the advantage of being double validated by the original population model in the growth cone and agent based simulations in the axon.

\end{block}

%------------------------------------------------
% reference block
\begin{block}{References}

\fontsize{16pt}{12pt}\selectfont
\nocite{craig2015pb, craig2012bj, devincentiis2020jn, kalil2005con, nedelec1997n13, raffa2023scdb, sanchez-huertas2021fmn}
\bibliographystyle{plain}
\bibliography{../proposal/axon_growth2.bib}

\end{block}

\end{column}
\separatorcolumn%

\end{columns}
\end{frame}
\end{document}
