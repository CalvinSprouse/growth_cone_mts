\documentclass[a4paper, 12pt]{article}

% import packages
\usepackage{authblk}

% title config
\title{Motor based microtubule sliding forces in axon growth cones}
\author{Calvin Sprouse}
\affil{PHYS 322, Winter 2024}
\date{2024 Februrary 27}

% bibliography config
\bibliographystyle{abbrv}

\begin{document}
\maketitle

\section*{Introduction}
% from cell to neuron to axon
Living cells house a complex internal symphony of functions mediated by proteins. These functions are preformed on the background of the cytoskeleton, a dynamic weave of filaments guided and locked by force generating proteins providing structure to the cell. The neuron is just one class of living cell adapted to function as a network signal transceiver or node. Nodes require connections and for this neurons develop two main protrusions from the cell body beginning as neurites. Most neurites become dendrites; these smaller branches support bidirectional transport of both signal and cargo. They also serve as targets for incoming network connections from other neurons. One neurite on each neuron becomes the axon; this longer branch is adapted to support outbound signaling from the cell body. The extent of neuronal protrusion varies by function over a magnitude scale of micrometers to meters. Regardless of distance, the axon grows following chemical cues and reaches out to dendrites forming the neuronal network.

% from axon to internal microtubule structure to growth cone
The axon, being uniquely designed for signal propagation, expresses uni\-que traits in the neuron context. Most notably is the emergent ordering of microtubule filaments in the axon \cite{nedelec1997n13}. Microtubules are dynamic assemblies of tubulin that serve as molecular highways. The plus-end of a microtubule (mt) is where most polymerization activity occurs. The polarity of an mt is its relative orientation in the context of the cell body; if an mts plus-end is further from the cell body than the minus-end, where polymerization activity tends not to occur, that mt is plus-end-out. Molecular motors are sensitive to polarity as each motor has a preference for walking toward the plus-end or minus-end. The polarity pattern then describes the spacial distribution of mt polarity and indicates the function of the region. It follows that dendrites express roughly even distributions of plus-end-out and minus-end-out mts as they are configured for bidirectional cargo transport. Axons, being outbound signal pathways, express a nearly exclusive plus-end-out pattern~\cite{nedelec1997n13}.

% but what does the growth cone do
Axon growth is lead by a region called the growth cone (gc). The gc is distinguished from the rest of the axon by a sharp decrease in mt density and a diffusion of the cytoplasm from the otherwise mostly cylindrical body of the axon. The gc houses the machinery for axon elongation. This includes sensing equipment for reading chemical guidance cues and force generating actomyosin network. The actomyosin network is a treadmill of actin filaments and myosin motors thought to be the primary driver in elongation~\cite{craig2012bj}. The treadmilling behavior is provided by leading end polymerization and trailing end depolymerization of actin~\cite{craig2012bj}. Myosin acts to hold the network together and recycle trailing end actin material to the leading end. It is currently understood that the function of mts is to convey signaling information from sensory equipment to the actomyosin network inducing a clutch-like response causing increase or decrease of adhesion to the substrate~\cite{craig2012bj,kalil2005con,sanchez-huertas2021fmn}.

In this proposal we aim to investigate the role of the mt array as a force generating mechanism of growth cone behavior. The forces acting on the gc are observed to be small, on the order of pN~\cite{raffa2023scdb,devincentiis2020jn}. Furthermore, it seems the mts protruding into the gc act to stabilize against contractile forces generated in the actomyosin network and by the gc membrane~\cite{raffa2023scdb}. We propose a computational study extending work done in Ref.~\cite{craig2015pb} to include mt force generation as a consequence of dynein motivated sorting action occurring in the axon body~\cite{nedelec1997n13}.

\section*{Model}

We will begin with Ref.~\cite{craig2015pb} which describes a computational adhesion clutch model in the context of the actomyosin network. In addition to the features of this model we will include protrusion of mts into the gc based on dynein polarity sorting action near the growth cone axon boundary. This will begin with an array of terminally contacting mts on which dynein binds stochastically. The effect of dynein will induce net protrusion as mts will be assumed to be stabilized by a crosslinked network of plus-end-out mts. This protruding mt will then act as a crosslinking site allowing motor binding to occur between the mt and the actomyosin network. We will then tune the model and compare the net force generated and subsequent growth rate to observational values~\cite{raffa2023scdb,devincentiis2020jn}.

% bibliography
\bibliography{axon_growth2}
\end{document}
